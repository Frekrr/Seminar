\section{Einleitung}

% und großer Dank an ChatGPT




Es besteht kein Zweifel daran, dass der aktuelle Finanzsektor trotz gro\ss er technologischer Entwicklungen 
immer noch ineffizient ist. In der traditionellen Bankenwelt ist das manuelle Abwickeln verschiedener 
Prozesse noch weit verbreitet. Von der Unterschrift auf einem Vertrag bis hin zur Veranlassung einer 
Überweisung - alle diese Prozesse erfordern Zeit und entsprechende Sorgfalt. Insbesondere die langsame 
Abwicklung von Verträgen und Zahlungen wirkt sich auf die Kundenzufriedenheit und die Wirtschaftlichkeit 
aus. Darüber hinaus stellt der Mensch als Fehlerquelle stets ein zusätzliches Risiko dar, das zu 
Verzögerungen und möglichen Kosten führen kann.

Heutzutage sind Banken- und mobile Bezahlsysteme wie Paypal in erster Linie zentral strukturiert, was 
bedeutet, dass das Vertrauen der Kunden in diese dritten Anbieter von entscheidender Bedeutung ist, 
insbesondere in Bezug auf die Datensicherheit. Des Weiteren stellt ein zentraler Knotenpunkt, in welchem
zwangsweise alle Datenflüsse zusammenflie\ss en und sich sämtliche Daten konzentrieren, ein attraktives 
Ziel für Cyberangriffe dar. Zusätzlich werfen unbekannte Datenflüsse und Funktionsweisen weitere 
Sicherheitsbedenken auf.


Die Revolutionierung traditioneller Finanzdienstleistungen durch innovative Technologien wie 
Blockchain und IoT öffnet Türen für eine Fülle von Möglichkeiten zur Effizienz- und Transparenzsteigerung.
%Auf der Suche nach Lösungen für diese Probleme zeigt die Kombination aus Blockchain-Technologie und 
%Internet der Dinge (IoT) ein enormes Potenzial. 
Die Blockchain kann die Automatisierung und Effizienzsteigerung der Prozesse ermöglichen und zudem durch
die Dezentralisierung viele Probleme von zentralisierten Systemen vermeiden. Währenddessen sorgt das IoT
für die Erhebung von Echtzeitdaten und verteilt diese im Netzwerk.
% nye
Trotz des vielversprechenden Potenzials, das diese Technologien bieten, gibt es jedoch 
auch rechtliche Hürden, die Berücksichtigung finden müssen. Es wird spannend sein zu sehen, wie sich 
Technologie und Gesetzgebung gemeinsam entwickeln, um den Finanzsektor effizienter, sicherer und 
kundenfreundlicher zu gestalten.

% Erneuerbare



% Blockchain hat neue Möglichkeiten für Kleinanleger / private Anleger mit sich gebracht.

% Industrie 4.0 wird immer relevanter um die Effizienz zu steigern, Prozesse zu Optimieren
% und weltweit konkurrenzfähig zu bleiben.

% Außerdem bieten sich neue Möglichkeiten um die Wirtschaft zu globaliesieren(?)  Lieferketten\dots 

% Einsatzgebiete für eine praktikable Integration von BC und IoT im Finanzsektor aufgezeigt. 
% Die bereits in der Praxis angewendet werden oder existierende Prozesse optimieren können.