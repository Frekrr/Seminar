\section{Blockchain}
In diesem Kapitel wird auf die revolutionären Vorteile der Blockchain 
eingegangen, durch die diese Technologie Einzug in den Finanzsektor gewonnen 
hat. Dem werden im Nachgang die einhergehenden Nachteile und Risiken gegenübergestellt. 

\subsection{Aufbau einer Blockchain}
\cite[p.~2]{pirafelnerblockchaintechnologie}
Um die Einsatzgebiete für eine Blockchain im Finanzsektor besser zu verstehen,
wird im Folgenden der Aufbau und die Komponenten grob dargestellt.
Im Block 1 wird zuerst jeweils ein Hash H1 und H2 für die Datenpunkte Transaktion 1 
und Transaktion 2 gebildet. Aus diesen Hashes wird dann ein gemeinsamer Hash H12 gebildet, 
der den Block Header 1 darstellt.
Analog zum Block Header 1 wird Block Header 2 erstellt. Zusätzlich enthält dieser eine
Referenz auf den vorigen Block Header 1 und ist somit der Kopf der Kette. Der Block Header 2
kann weiterführend im Block Haeder 3 eines dritten Blocks referenziert werden. 
\ref{fig:BC_Aufbau}

\begin{figure}[h] %mit b und p statt h probieren
    \includegraphics[width=\columnwidth]{BC_Aufbau.png}
    \caption{Aufbau einer Blockchain}
    \label{fig:BC_Aufbau}
\end{figure}

\cite[p.~17f]{fill2020aufbau}

\subsection{}