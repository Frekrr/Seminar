\textbf{Abstract} 

\vspace{1.5cm}


Dieser Artikel bietet eine umfassende Analyse der Anwendung und des Potenzials der 
Blockchain-Technologie im Finanzsektor, insbesondere bei Transaktionen, Smart Contracts und 
der Tokenisierung von Assets. 
Dargestellt wird, wie Blockchain-Systeme durch das Nutzen von 
Konsensalgorithmen die Sicherheit und Nachvollziehbarkeit von Transaktionen gewährleisten 
und somit traditionelle, kostenintensive Intermediäre umgehen. 
Außerdem werden die
Möglichkeiten von Smart Contracts erörtert, die sowohl als digitale Verträge fungieren, 
die automatisch ausgeführt werden, als auch zur Risiko- und Betrugsprävention in 
Versicherungsfällen dienen können. Des Weiteren werden die Vorteile der Tokenisierung von 
Assets, einschließlich der Verbesserung der Liquidität und Zugänglichkeit von 
Vermögenswerten, erläutert. Zusätzlich wird die Synergie zwischen Blockchain und Internet 
of Things (IoT) untersucht, wobei auf die möglichen Vorteile für 
Lieferkettenfinanzierung und Risikomanagement hingewiesen wird. Auch auf die 
datenschutzrechtlichen Risiken wird eingegangen und erforderliche Einhaltungen aufgezeigt.
Die Herausforderungen in Bezug auf Nachhaltigkeit und 
Energieverbrauch von Blockchains werden aufgezeigt und eine Lösung anhand eines 
Vergleichs verschiedener Konsensalgorithmen geboten. Insgesamt liefert der Artikel wertvolle 
Einblicke in die aktuelle Nutzung 
und das zukünftige Potenzial der Blockchain-Technologie im Finanzsektor.