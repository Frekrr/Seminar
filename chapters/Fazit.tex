\section{Fazit}

Eine Integration der Blockchain-Technologie bietet äußerst positive Möglichkeiten, um
Abläufe im Finanzsektor zu automatisieren und für alle Beteiligten zu optimieren.

Im Allgemeinen bietet der Aufbau einer Blockchain eine koherente Nachvollziehbarkeit aller 
Transaktionen und sichert die Integrität und Korrektheit der Daten. 
Aufgrund dessen kann sie eine Alternative für die Finanzbuchhaltung darstellen und bietet 
einen Schutz vor Bilanzfälschungen, da sonst die Integrität nicht gegeben wäre.
Des Weiteren sorgt die Dezentralisierung der Blockchain für eine hohe Robustheit und 
Transparenz der Systeme.
Ausfälle von einzelnen Rechnerknoten können ohne einen Mehraufwand von anderen Knoten
kompensiert werden und insgesamt kann auf zentrale Datenspeicher verzichtet werden.
\cite[p.~70]{fill2020blockchain}
Zudem sorgen die Eigenschaften eines Blockchain-basierten Systems dafür, dass es ein
unattraktiveres Ziel von Cyberangriffen wird und dadurch finanzielle Schäden verhindert
werden. Dies ist ein enormer Vorteil im Finanzsektor, da dieser immer ein lukratives Ziel 
für Hacker sein wird.
\newline
Die Nutzung von Smart Contracts sorgt für bemerkenswerte Effizienssteigerungen in
Geschäftsprozessen. 
Aufgrund der Transparenz durch definierte Abläufe in den Vertragsbestimmungen können 
rechtliche Streitigkeiten reduziert werden. Dadurch werden Unterbrechungen zur Klärung von
Rechtsfragen verringert und Anwaltskosten reduziert. Eine zeitliche Abschätzung von 
Geschäftsprozessen wird hierdurch planbarer.
Jedoch müssen die Vertragsbestimmungen vorab juristisch abgesichert sein, sodass der Smart 
Contract seine rechtliche Verbindlichkeit behält. Dies kann vor allem bei internationalen
Verträgen problematisch sein, da mehrere örtliche Regelungen berücksichtigt werden müssen.
Dieser Mehraufwand in der Vorbereitung ist lohnenswert, da sowohl Fehlerquellen, 
als auch Bearbeitungszeiten und Gebührenkosten verringert werden.
\newline
Die Blockchain ist demnach eine wünschenswerte Ergänzung im Finanzsektor, die als 
vertrauenswürdiger Dienst zentralisierte Systeme ersetzen kann.
Bei der Nutzung von bereits bestehenden Blockchains kann jedoch eine fehlende zentrale 
Instanz, die die Weiterentwicklung der Technologie koordiniert, für Probleme in der 
Abwärtskompatibilität sorgen. In dem Fall muss auf die Entwickler-Community vertraut werden,
wodurch keine Planungsstabilität gegeben ist.
\cite[p.~70f]{fill2020blockchain}










Darüber hinaus ist die Kritik an der Nachhaltigkeit der Ressourcenverwendung in 
Blockchain-Technologien hervorzuheben. Verschiedene Konsensalgorithmen wie Proof-of-Stake 
oder Proof-of-Energy-Efficiency können jedoch zur Reduzierung des Energieverbrauchs 
beitragen. 
\newline
Insgesamt bietet die Blockchain-Technologie dem Finanzsektor und darüber hinaus bedeutende 
Vorteile, verlangt jedoch nach einem verantwortungsbewussten Umgang mit Datensicherheit, 
Datenschutz und Energieverbrauch.