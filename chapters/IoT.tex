\section{Blockchain und Internet of Things}
Die Blockchain-Technologie bringt bereits einige Vorteile für Finanzsektor mit sich.
In der Kombination mit Internet of Things (IoT) lassen sich einige der bereits erörterten Einsatzgebiete
für breitere Anwendungsmöglichkeiten erweitern.

\subsection{Anwendungsmöglichkeiten/Synergien mit Internet of Things}

\subsubsection{Lieferkettenfinanzierung}
Integriert man IoT-Sensoren mit einer Blockchain-Struktur, können in der Lieferkette Zeit und Kosten bei
jeder Übergabe eingespart werden.
% Asset Tracking
Die IoT-Technologie schafft die Möglichkeit für Asset Tracking, um den Zustand von physischen Assets
kontinuierlich einsehen zu können. Durch das Anbringen von internetfähigen Sensoren an 
Liefercontainern kann so der Status des Transports vollumfänglich transparent gestaltet
werden und in Echtzeit abgefragt werden. 
Informationen die hierbei von Interesse sind, wären der aktuelle Standort der Lieferung und
die vorraussichtliche Ankunftszeit. Erweiternd dazu kann für empfindliche und 
verderbliche Waren die Temperatur, Luftfeuchtigkeit oder starke Erschütterungen auf dem 
Transportweg aufgezeichnet werden.
% Speichern und Smart Contract
Die Blockchain-Technologie bietet die Funktionalität, die erfassten Informationen in Blöcken
zu speichern. Damit fungiert die Blockchain wie ein Transportbericht der gesamten Lieferkette und 
sichert nebenbei die Integrität und Richtigkeit der Daten, sofern keine Sensorprobleme vorliegen. 
Zusätzlich kann bei der Zustellung der Lieferung ein Smart Contract die Zahlung automatisch durchführen.
So werden der Zeitaufwand und die Kosten für Qualitätsprüfungen reduziert und vor allem verderbliche 
Waren können schneller weiterverarbeitet werden.
\cite[p.~169f]{chowdhary2025smart}


\subsubsection{Risiko-Management}
Der kontinuierliche Datenstrom von IoT-Sensoren kann zur Risikobewertung rangezogen werden.
Die Auswertung dieser Daten kann im Nachgang zur Eingrenzung potentieller Schäden 
verwendet werden.
Versicherungsunternehmen können die Daten aus dem Auto verwenden, sofern es beispielsweise
eine Car2Car-Kommunikation besitzt, um die Beiträge an das individuelle Fahrverhalten
anzupassen. 
Wenn die Sensoren oft Geschwindigkeitsüberschreitungen oder nicht-einhalten von Mindestabständen
verzeichnen, kann der Beitrag für die einzelne Person erhöht werden. Auf der Gegenseite kann
sicheres Fahren mit niedrigeren Beiträgen belohnt werden.
\cite[p.~169f]{chowdhary2025smart}

\cite[p.~347ff]{Zhang2021Research}
